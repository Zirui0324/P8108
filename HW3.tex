% Options for packages loaded elsewhere
\PassOptionsToPackage{unicode}{hyperref}
\PassOptionsToPackage{hyphens}{url}
%
\documentclass[
]{article}
\usepackage{amsmath,amssymb}
\usepackage{iftex}
\ifPDFTeX
  \usepackage[T1]{fontenc}
  \usepackage[utf8]{inputenc}
  \usepackage{textcomp} % provide euro and other symbols
\else % if luatex or xetex
  \usepackage{unicode-math} % this also loads fontspec
  \defaultfontfeatures{Scale=MatchLowercase}
  \defaultfontfeatures[\rmfamily]{Ligatures=TeX,Scale=1}
\fi
\usepackage{lmodern}
\ifPDFTeX\else
  % xetex/luatex font selection
\fi
% Use upquote if available, for straight quotes in verbatim environments
\IfFileExists{upquote.sty}{\usepackage{upquote}}{}
\IfFileExists{microtype.sty}{% use microtype if available
  \usepackage[]{microtype}
  \UseMicrotypeSet[protrusion]{basicmath} % disable protrusion for tt fonts
}{}
\makeatletter
\@ifundefined{KOMAClassName}{% if non-KOMA class
  \IfFileExists{parskip.sty}{%
    \usepackage{parskip}
  }{% else
    \setlength{\parindent}{0pt}
    \setlength{\parskip}{6pt plus 2pt minus 1pt}}
}{% if KOMA class
  \KOMAoptions{parskip=half}}
\makeatother
\usepackage{xcolor}
\usepackage[margin=1in]{geometry}
\usepackage{color}
\usepackage{fancyvrb}
\newcommand{\VerbBar}{|}
\newcommand{\VERB}{\Verb[commandchars=\\\{\}]}
\DefineVerbatimEnvironment{Highlighting}{Verbatim}{commandchars=\\\{\}}
% Add ',fontsize=\small' for more characters per line
\usepackage{framed}
\definecolor{shadecolor}{RGB}{248,248,248}
\newenvironment{Shaded}{\begin{snugshade}}{\end{snugshade}}
\newcommand{\AlertTok}[1]{\textcolor[rgb]{0.94,0.16,0.16}{#1}}
\newcommand{\AnnotationTok}[1]{\textcolor[rgb]{0.56,0.35,0.01}{\textbf{\textit{#1}}}}
\newcommand{\AttributeTok}[1]{\textcolor[rgb]{0.13,0.29,0.53}{#1}}
\newcommand{\BaseNTok}[1]{\textcolor[rgb]{0.00,0.00,0.81}{#1}}
\newcommand{\BuiltInTok}[1]{#1}
\newcommand{\CharTok}[1]{\textcolor[rgb]{0.31,0.60,0.02}{#1}}
\newcommand{\CommentTok}[1]{\textcolor[rgb]{0.56,0.35,0.01}{\textit{#1}}}
\newcommand{\CommentVarTok}[1]{\textcolor[rgb]{0.56,0.35,0.01}{\textbf{\textit{#1}}}}
\newcommand{\ConstantTok}[1]{\textcolor[rgb]{0.56,0.35,0.01}{#1}}
\newcommand{\ControlFlowTok}[1]{\textcolor[rgb]{0.13,0.29,0.53}{\textbf{#1}}}
\newcommand{\DataTypeTok}[1]{\textcolor[rgb]{0.13,0.29,0.53}{#1}}
\newcommand{\DecValTok}[1]{\textcolor[rgb]{0.00,0.00,0.81}{#1}}
\newcommand{\DocumentationTok}[1]{\textcolor[rgb]{0.56,0.35,0.01}{\textbf{\textit{#1}}}}
\newcommand{\ErrorTok}[1]{\textcolor[rgb]{0.64,0.00,0.00}{\textbf{#1}}}
\newcommand{\ExtensionTok}[1]{#1}
\newcommand{\FloatTok}[1]{\textcolor[rgb]{0.00,0.00,0.81}{#1}}
\newcommand{\FunctionTok}[1]{\textcolor[rgb]{0.13,0.29,0.53}{\textbf{#1}}}
\newcommand{\ImportTok}[1]{#1}
\newcommand{\InformationTok}[1]{\textcolor[rgb]{0.56,0.35,0.01}{\textbf{\textit{#1}}}}
\newcommand{\KeywordTok}[1]{\textcolor[rgb]{0.13,0.29,0.53}{\textbf{#1}}}
\newcommand{\NormalTok}[1]{#1}
\newcommand{\OperatorTok}[1]{\textcolor[rgb]{0.81,0.36,0.00}{\textbf{#1}}}
\newcommand{\OtherTok}[1]{\textcolor[rgb]{0.56,0.35,0.01}{#1}}
\newcommand{\PreprocessorTok}[1]{\textcolor[rgb]{0.56,0.35,0.01}{\textit{#1}}}
\newcommand{\RegionMarkerTok}[1]{#1}
\newcommand{\SpecialCharTok}[1]{\textcolor[rgb]{0.81,0.36,0.00}{\textbf{#1}}}
\newcommand{\SpecialStringTok}[1]{\textcolor[rgb]{0.31,0.60,0.02}{#1}}
\newcommand{\StringTok}[1]{\textcolor[rgb]{0.31,0.60,0.02}{#1}}
\newcommand{\VariableTok}[1]{\textcolor[rgb]{0.00,0.00,0.00}{#1}}
\newcommand{\VerbatimStringTok}[1]{\textcolor[rgb]{0.31,0.60,0.02}{#1}}
\newcommand{\WarningTok}[1]{\textcolor[rgb]{0.56,0.35,0.01}{\textbf{\textit{#1}}}}
\usepackage{graphicx}
\makeatletter
\def\maxwidth{\ifdim\Gin@nat@width>\linewidth\linewidth\else\Gin@nat@width\fi}
\def\maxheight{\ifdim\Gin@nat@height>\textheight\textheight\else\Gin@nat@height\fi}
\makeatother
% Scale images if necessary, so that they will not overflow the page
% margins by default, and it is still possible to overwrite the defaults
% using explicit options in \includegraphics[width, height, ...]{}
\setkeys{Gin}{width=\maxwidth,height=\maxheight,keepaspectratio}
% Set default figure placement to htbp
\makeatletter
\def\fps@figure{htbp}
\makeatother
\setlength{\emergencystretch}{3em} % prevent overfull lines
\providecommand{\tightlist}{%
  \setlength{\itemsep}{0pt}\setlength{\parskip}{0pt}}
\setcounter{secnumdepth}{-\maxdimen} % remove section numbering
\usepackage{booktabs}
\usepackage{longtable}
\usepackage{array}
\usepackage{multirow}
\usepackage{wrapfig}
\usepackage{float}
\usepackage{colortbl}
\usepackage{pdflscape}
\usepackage{tabu}
\usepackage{threeparttable}
\usepackage{threeparttablex}
\usepackage[normalem]{ulem}
\usepackage{makecell}
\usepackage{xcolor}
\ifLuaTeX
  \usepackage{selnolig}  % disable illegal ligatures
\fi
\IfFileExists{bookmark.sty}{\usepackage{bookmark}}{\usepackage{hyperref}}
\IfFileExists{xurl.sty}{\usepackage{xurl}}{} % add URL line breaks if available
\urlstyle{same}
\hypersetup{
  hidelinks,
  pdfcreator={LaTeX via pandoc}}

\author{}
\date{\vspace{-2.5em}}

\begin{document}

\hypertarget{question-3}{%
\section{Question 3}\label{question-3}}

\begin{Shaded}
\begin{Highlighting}[]
\NormalTok{data }\OtherTok{=}\NormalTok{ ovarian }\SpecialCharTok{\%\textgreater{}\%} \FunctionTok{mutate}\NormalTok{(}\AttributeTok{fustat =} \FunctionTok{as.factor}\NormalTok{(fustat), }\AttributeTok{rx =} \FunctionTok{as.factor}\NormalTok{(rx))}
\end{Highlighting}
\end{Shaded}

\hypertarget{create-life-time-stratified-by-rx}{%
\subsubsection{1. Create life-time stratified by
rx}\label{create-life-time-stratified-by-rx}}

\begin{Shaded}
\begin{Highlighting}[]
\CommentTok{\# Create life{-}table stratified by rx}
\NormalTok{lt1 }\OtherTok{=} \FunctionTok{lifetab2}\NormalTok{(}\FunctionTok{Surv}\NormalTok{(futime, fustat }\SpecialCharTok{==} \DecValTok{1}\NormalTok{) }\SpecialCharTok{\textasciitilde{}} \DecValTok{1}\NormalTok{, }
\NormalTok{               data[data}\SpecialCharTok{$}\NormalTok{rx }\SpecialCharTok{==} \DecValTok{1}\NormalTok{, ], }
               \AttributeTok{breaks=} \FunctionTok{seq}\NormalTok{(}\DecValTok{0}\NormalTok{, }\DecValTok{1300}\NormalTok{, }\DecValTok{100}\NormalTok{)) }\CommentTok{\#rx=1}
\NormalTok{lt2 }\OtherTok{=} \FunctionTok{lifetab2}\NormalTok{(}\FunctionTok{Surv}\NormalTok{(futime, fustat }\SpecialCharTok{==} \DecValTok{1}\NormalTok{) }\SpecialCharTok{\textasciitilde{}} \DecValTok{1}\NormalTok{, }
\NormalTok{               data[data}\SpecialCharTok{$}\NormalTok{rx }\SpecialCharTok{==} \DecValTok{2}\NormalTok{, ], }
               \AttributeTok{breaks=} \FunctionTok{seq}\NormalTok{(}\DecValTok{0}\NormalTok{, }\DecValTok{1300}\NormalTok{, }\DecValTok{100}\NormalTok{)) }\CommentTok{\#rx=2}
\FunctionTok{options}\NormalTok{(}\AttributeTok{digits=}\DecValTok{2}\NormalTok{)}
\FunctionTok{print}\NormalTok{(lt1) }\CommentTok{\# life{-}table for treatment 1}
\end{Highlighting}
\end{Shaded}

\begin{verbatim}
##           tstart tstop nsubs nlost nrisk nevent surv     pdf hazard se.surv
## 0-100          0   100    13     0  13.0      1 1.00 0.00077 0.0008   0.000
## 100-200      100   200    12     0  12.0      2 0.92 0.00154 0.0018   0.074
## 200-300      200   300    10     0  10.0      1 0.77 0.00077 0.0011   0.117
## 300-400      300   400     9     0   9.0      1 0.69 0.00077 0.0012   0.128
## 400-500      400   500     8     2   7.0      1 0.62 0.00088 0.0015   0.135
## 500-600      500   600     5     0   5.0      0 0.53 0.00000 0.0000   0.141
## 600-700      600   700     5     0   5.0      1 0.53 0.00105 0.0022   0.141
## 700-800      700   800     4     0   4.0      0 0.42 0.00000 0.0000   0.147
## 800-900      800   900     4     2   3.0      0 0.42 0.00000 0.0000   0.147
## 900-1000     900  1000     2     0   2.0      0 0.42 0.00000 0.0000   0.147
## 1000-1100   1000  1100     2     1   1.5      0 0.42 0.00000 0.0000   0.147
## 1100-1200   1100  1200     1     1   0.5      0 0.42 0.00000 0.0000   0.147
## 1200-1300   1200  1300     0     0   0.0      0 0.42     NaN    NaN   0.147
## 1300-Inf    1300   Inf     0     0   0.0      0  NaN      NA     NA     NaN
##            se.pdf se.hazard
## 0-100     0.00074    0.0008
## 100-200   0.00100    0.0013
## 200-300   0.00074    0.0011
## 300-400   0.00074    0.0012
## 400-500   0.00084    0.0015
## 500-600       NaN       NaN
## 600-700   0.00099    0.0022
## 700-800       NaN       NaN
## 800-900       NaN       NaN
## 900-1000      NaN       NaN
## 1000-1100     NaN       NaN
## 1100-1200     NaN       NaN
## 1200-1300     NaN       NaN
## 1300-Inf       NA        NA
\end{verbatim}

\begin{Shaded}
\begin{Highlighting}[]
\FunctionTok{print}\NormalTok{(lt2) }\CommentTok{\# life{-}table for treatment 2}
\end{Highlighting}
\end{Shaded}

\begin{verbatim}
##           tstart tstop nsubs nlost nrisk nevent surv     pdf hazard se.surv
## 0-100          0   100    13     0  13.0      0 1.00 0.00000 0.0000    0.00
## 100-200      100   200    13     0  13.0      0 1.00 0.00000 0.0000    0.00
## 200-300      200   300    13     0  13.0      0 1.00 0.00000 0.0000    0.00
## 300-400      300   400    13     1  12.5      2 1.00 0.00160 0.0017    0.00
## 400-500      400   500    10     1   9.5      2 0.84 0.00177 0.0024    0.10
## 500-600      500   600     7     0   7.0      1 0.66 0.00095 0.0015    0.14
## 600-700      600   700     6     0   6.0      0 0.57 0.00000 0.0000    0.15
## 700-800      700   800     6     3   4.5      0 0.57 0.00000 0.0000    0.15
## 800-900      800   900     3     0   3.0      0 0.57 0.00000 0.0000    0.15
## 900-1000     900  1000     3     0   3.0      0 0.57 0.00000 0.0000    0.15
## 1000-1100   1000  1100     3     0   3.0      0 0.57 0.00000 0.0000    0.15
## 1100-1200   1100  1200     3     1   2.5      0 0.57 0.00000 0.0000    0.15
## 1200-1300   1200  1300     2     2   1.0      0 0.57 0.00000 0.0000    0.15
## 1300-Inf    1300   Inf     0     0   0.0      0 0.57      NA     NA    0.15
##           se.pdf se.hazard
## 0-100        NaN       NaN
## 100-200      NaN       NaN
## 200-300      NaN       NaN
## 300-400   0.0010    0.0012
## 400-500   0.0011    0.0017
## 500-600   0.0009    0.0015
## 600-700      NaN       NaN
## 700-800      NaN       NaN
## 800-900      NaN       NaN
## 900-1000     NaN       NaN
## 1000-1100    NaN       NaN
## 1100-1200    NaN       NaN
## 1200-1300    NaN       NaN
## 1300-Inf      NA        NA
\end{verbatim}

\hypertarget{plot-harzard-function-by-rx}{%
\subsubsection{2. Plot harzard function by
rx}\label{plot-harzard-function-by-rx}}

\begin{Shaded}
\begin{Highlighting}[]
\FunctionTok{ggplot}\NormalTok{(lt1, }\FunctionTok{aes}\NormalTok{(}\AttributeTok{x =}\NormalTok{ tstart, }\AttributeTok{xend =}\NormalTok{ tstop, }\AttributeTok{y =}\NormalTok{ hazard, }\AttributeTok{yend =}\NormalTok{ hazard)) }\SpecialCharTok{+}
  \FunctionTok{geom\_segment}\NormalTok{() }\SpecialCharTok{+}
  \FunctionTok{labs}\NormalTok{(}\AttributeTok{x =} \StringTok{"t"}\NormalTok{, }\AttributeTok{y =} \StringTok{"h(t)"}\NormalTok{) }\SpecialCharTok{+}
  \FunctionTok{ggtitle}\NormalTok{(}\StringTok{"Hazard function for treatment 1"}\NormalTok{) }\SpecialCharTok{+}
  \FunctionTok{theme\_minimal}\NormalTok{() }\CommentTok{\#rx=1}
\end{Highlighting}
\end{Shaded}

\includegraphics{HW3_files/figure-latex/unnamed-chunk-3-1.pdf}

\begin{Shaded}
\begin{Highlighting}[]
\FunctionTok{ggplot}\NormalTok{(lt2, }\FunctionTok{aes}\NormalTok{(}\AttributeTok{x =}\NormalTok{ tstart, }\AttributeTok{xend =}\NormalTok{ tstop, }\AttributeTok{y =}\NormalTok{ hazard, }\AttributeTok{yend =}\NormalTok{ hazard)) }\SpecialCharTok{+}
  \FunctionTok{geom\_segment}\NormalTok{() }\SpecialCharTok{+}
  \FunctionTok{labs}\NormalTok{(}\AttributeTok{x =} \StringTok{"t"}\NormalTok{, }\AttributeTok{y =} \StringTok{"h(t)"}\NormalTok{) }\SpecialCharTok{+}
  \FunctionTok{ggtitle}\NormalTok{(}\StringTok{"Hazard function for treatment 2"}\NormalTok{)}\SpecialCharTok{+}
  \FunctionTok{theme\_minimal}\NormalTok{() }\CommentTok{\#rx=2}
\end{Highlighting}
\end{Shaded}

\includegraphics{HW3_files/figure-latex/unnamed-chunk-3-2.pdf}

\hypertarget{plot-k-m-survival-function-by-rx}{%
\subsubsection{3. Plot K-M survival function by
rx}\label{plot-k-m-survival-function-by-rx}}

\begin{Shaded}
\begin{Highlighting}[]
\NormalTok{mfit }\OtherTok{=} \FunctionTok{survfit}\NormalTok{(}\FunctionTok{Surv}\NormalTok{(futime, fustat }\SpecialCharTok{==} \DecValTok{1}\NormalTok{) }\SpecialCharTok{\textasciitilde{}}\NormalTok{ rx, data)}
\FunctionTok{plot}\NormalTok{(mfit, }\AttributeTok{ylab=}\StringTok{"S(t)"}\NormalTok{, }\AttributeTok{xlab=}\StringTok{"Time in days"}\NormalTok{,}
     \AttributeTok{main =} \StringTok{"Kaplan−Meier estimates of treatment{-}specific survival"}\NormalTok{)}
\end{Highlighting}
\end{Shaded}

\includegraphics{HW3_files/figure-latex/unnamed-chunk-4-1.pdf}

\hypertarget{median-for-each-treatment-group}{%
\subsubsection{4. Median for each treatment
group}\label{median-for-each-treatment-group}}

\begin{itemize}
\tightlist
\item
  rx 1: median = 638
\item
  rx 2: median doesn't exist
\end{itemize}

\begin{Shaded}
\begin{Highlighting}[]
\FunctionTok{print}\NormalTok{(mfit)}
\end{Highlighting}
\end{Shaded}

\begin{verbatim}
## Call: survfit(formula = Surv(futime, fustat == 1) ~ rx, data = data)
## 
##       n events median 0.95LCL 0.95UCL
## rx=1 13      7    638     268      NA
## rx=2 13      5     NA     475      NA
\end{verbatim}

\hypertarget{compare-survival-function-estimations-between-k-m-and-f-h-methods}{%
\subsubsection{5. Compare survival function estimations between K-M and
F-H
methods}\label{compare-survival-function-estimations-between-k-m-and-f-h-methods}}

\begin{Shaded}
\begin{Highlighting}[]
\NormalTok{hfit }\OtherTok{=} \FunctionTok{survfit}\NormalTok{(}\FunctionTok{Surv}\NormalTok{(futime, fustat }\SpecialCharTok{==} \DecValTok{1}\NormalTok{) }\SpecialCharTok{\textasciitilde{}}\NormalTok{ rx, }\AttributeTok{type =} \StringTok{"fleming{-}harrington"}\NormalTok{, data)}
\FunctionTok{print}\NormalTok{(hfit)}
\end{Highlighting}
\end{Shaded}

\begin{verbatim}
## Call: survfit(formula = Surv(futime, fustat == 1) ~ rx, data = data, 
##     type = "fleming-harrington")
## 
##       n events median 0.95LCL 0.95UCL
## rx=1 13      7    638     268      NA
## rx=2 13      5     NA     475      NA
\end{verbatim}

\begin{Shaded}
\begin{Highlighting}[]
\FunctionTok{summary}\NormalTok{(mfit)}
\end{Highlighting}
\end{Shaded}

\begin{verbatim}
## Call: survfit(formula = Surv(futime, fustat == 1) ~ rx, data = data)
## 
##                 rx=1 
##  time n.risk n.event survival std.err lower 95% CI upper 95% CI
##    59     13       1    0.923  0.0739        0.789        1.000
##   115     12       1    0.846  0.1001        0.671        1.000
##   156     11       1    0.769  0.1169        0.571        1.000
##   268     10       1    0.692  0.1280        0.482        0.995
##   329      9       1    0.615  0.1349        0.400        0.946
##   431      8       1    0.538  0.1383        0.326        0.891
##   638      5       1    0.431  0.1467        0.221        0.840
## 
##                 rx=2 
##  time n.risk n.event survival std.err lower 95% CI upper 95% CI
##   353     13       1    0.923  0.0739        0.789        1.000
##   365     12       1    0.846  0.1001        0.671        1.000
##   464      9       1    0.752  0.1256        0.542        1.000
##   475      8       1    0.658  0.1407        0.433        1.000
##   563      7       1    0.564  0.1488        0.336        0.946
\end{verbatim}

\begin{Shaded}
\begin{Highlighting}[]
\FunctionTok{summary}\NormalTok{(hfit)}
\end{Highlighting}
\end{Shaded}

\begin{verbatim}
## Call: survfit(formula = Surv(futime, fustat == 1) ~ rx, data = data, 
##     type = "fleming-harrington")
## 
##                 rx=1 
##  time n.risk n.event survival std.err lower 95% CI upper 95% CI
##    59     13       1    0.926  0.0712        0.796        1.000
##   115     12       1    0.852  0.0966        0.682        1.000
##   156     11       1    0.778  0.1131        0.585        1.000
##   268     10       1    0.704  0.1242        0.498        0.995
##   329      9       1    0.630  0.1313        0.419        0.948
##   431      8       1    0.556  0.1351        0.345        0.895
##   638      5       1    0.455  0.1433        0.246        0.843
## 
##                 rx=2 
##  time n.risk n.event survival std.err lower 95% CI upper 95% CI
##   353     13       1    0.926  0.0712        0.796        1.000
##   365     12       1    0.852  0.0966        0.682        1.000
##   464      9       1    0.762  0.1210        0.558        1.000
##   475      8       1    0.673  0.1359        0.453        1.000
##   563      7       1    0.583  0.1443        0.359        0.947
\end{verbatim}

\begin{Shaded}
\begin{Highlighting}[]
\NormalTok{mfit }\SpecialCharTok{\%\textgreater{}\%} \FunctionTok{autoplot}\NormalTok{() }\SpecialCharTok{+} \FunctionTok{ylab}\NormalTok{(}\StringTok{"S(t)"}\NormalTok{) }\SpecialCharTok{+} \FunctionTok{xlab}\NormalTok{(}\StringTok{"Time"}\NormalTok{)}
\end{Highlighting}
\end{Shaded}

\includegraphics{HW3_files/figure-latex/unnamed-chunk-6-1.pdf}

\begin{Shaded}
\begin{Highlighting}[]
\NormalTok{hfit }\SpecialCharTok{\%\textgreater{}\%} \FunctionTok{autoplot}\NormalTok{() }\SpecialCharTok{+} \FunctionTok{ylab}\NormalTok{(}\StringTok{"S(t)"}\NormalTok{) }\SpecialCharTok{+} \FunctionTok{xlab}\NormalTok{(}\StringTok{"Time"}\NormalTok{)}
\end{Highlighting}
\end{Shaded}

\includegraphics{HW3_files/figure-latex/unnamed-chunk-6-2.pdf}

The two methods give the same median estimations. Generally the F-H
method gave higher survival rates and lower se, which could be because
the di/ni rate is rather high in this dataset.

The survival rate of patients in treatment 1 is lower than those in
treatment 2, especially in the first 300 days,indicating treatment 2
could potentially be a better treatment. In treatment 1 group, less than
half the patients survived beyond 700 days, while in treatment 2 group,
more than 57\% patients survived beyond 1200 days.

\end{document}
